\documentclass[twocolumn]{report}
\usepackage{listings}
\usepackage{color}
\usepackage{xcolor}
\definecolor{javared}{rgb}{0.6,0,0} % for strings
\definecolor{javagreen}{rgb}{0.25,0.5,0.35} % comments
\definecolor{javapurple}{rgb}{0.5,0,0.35} % keywords
\definecolor{javadocblue}{rgb}{0.25,0.35,0.75} % javadoc
 
\lstset{language=Java,
breaklines=true,
basicstyle=\ttfamily\scriptsize,
keywordstyle=\color{javapurple}\bfseries,
stringstyle=\color{javared},
commentstyle=\color{javagreen},
morecomment=[s][\color{javadocblue}]{/**}{*/},
tabsize=4,
showspaces=false,
frame=lrtb,
showstringspaces=false}

\begin{document}
\title{Java Lambda Expression Source to Source Compiler}
\author{Kurt McAlpine}
\date{24 April 2015}
\maketitle

\chapter{Introduction}
An example use of my implementation of lambda expressions in java is to
instantiate an anonymous classes with exactly one method to be overriden. Using
lambda expressions it makes the code much cleaner and more concise.

An example use of this would be to instantiate an instance of a Runnable class
and override the run method.\cite{Ierusalimschy:2007:EL:1238844.1238846}

In the Java standard library there is this interface
\begin{lstlisting}
interface Runnable {
  void run();
}
\end{lstlisting}
With Java Lambda expressions instatiating an anonymous class in more consice.

No lambda expressions:
\begin{lstlisting}
Runnable r = new Runnable() {
	@Override
	public void run() {
		System.out.println("run!");
	}
};
r.run();
\end{lstlisting}
Using lambda expression:
\begin{lstlisting}
Runnable r = () -> { System.out.println("run!"); };    	
r.run();
\end{lstlisting}

\bibliographystyle{ieeetr}
\bibliography{references}
\end{document}
