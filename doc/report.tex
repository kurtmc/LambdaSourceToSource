\documentclass[twocolumn,notitlepage]{report}
\usepackage{hyperref}
\usepackage{listings}
\usepackage{color}
\usepackage{xcolor}
\definecolor{javared}{rgb}{0.6,0,0} % for strings
\definecolor{javagreen}{rgb}{0.25,0.5,0.35} % comments
\definecolor{javapurple}{rgb}{0.5,0,0.35} % keywords
\definecolor{javadocblue}{rgb}{0.25,0.35,0.75} % javadoc

% Change bibliography style
\usepackage{etoolbox}
\patchcmd{\thebibliography}{\chapter*}{\section*}{}{}
\renewcommand\bibname{References}

% No indentation
\setlength\parindent{0pt}
 
\lstset{language=Java,
breaklines=true,
basicstyle=\ttfamily\scriptsize,
keywordstyle=\color{javapurple}\bfseries,
stringstyle=\color{javared},
commentstyle=\color{javagreen},
morecomment=[s][\color{javadocblue}]{/**}{*/},
tabsize=4,
showspaces=false,
frame=lrtb,
showstringspaces=false}

\begin{document}
\title{Java Lambda Expression Source to Source Compiler}
\author{Kurt McAlpine}
\date{24 April 2015}
\maketitle

\section*{Introduction}
For this assignment I have decided to implement a feature similar to Java 8
lambda expressions. These expressions are used in place of anonymous class
instantiations.

An example use of my implementation of lambda expressions in java is to
instantiate an anonymous class with exactly one method to be overridden. By
lambda expressions, it makes the code much cleaner and more
concise.\cite{website:love-and-hate-for-java-8}

The following is an example use of my implementation of lambda expressions.

In the Java standard library there is the following interface:
\begin{lstlisting}
interface Runnable {
  void run();
}
\end{lstlisting}

Instantiating an anonymous instance of Runnable:
\begin{lstlisting}
Runnable r = new Runnable() {
	@Override
	public void run() {
		System.out.println("run!");
	}
};
r.run();
\end{lstlisting}
Now using lambda expression:
\begin{lstlisting}
Runnable r = () -> { System.out.println("run!"); };    	
r.run();
\end{lstlisting}

The syntax of a lambda expression starts with a list of parameters in
parenthesis. For example: {\ttfamily(int a, int b, String s)} , followed by a
new keyword ``{\ttfamily->}'' and then a series of statements surrounded by
braces. In the above example only one statement is used but there may be
multiple statements inside the braces.

\section*{Why Java Lambda Expressions are Important}
Currently (in Java 1.7 and below) programmers need to write a large amount of
boiler plate code to instantiate an anonymous class. It takes around 3 lines to
create an anonymous instance of Runnable that overrides the run() method with a
method body of just one line. There is clearly a need for improvement.

\section*{How Lambda Expressions Address the Problematic Issues that Programmers Face}
An example of a piece of Java code that requires a lot of boiler plate code is
in Android development. This can be exemplified when you are attaching a click
listener to a view such as a button. Therefore, you may have some code that
looks like this:
\begin{lstlisting}
Button myButton = (Button) findViewById(R.id.my_button);
myButton.setOnClickListener(
new View.OnClickListener() {
	@Override
	public void onClick(View v) {
		doAction(v);
	}
});
\end{lstlisting}
However, through using lambda expressions we can make this code less verbose
and therefore more readable and concise. Here is the above code rewritten to
use lambda expressions:
\newpage
\begin{lstlisting}
Button myButton = (Button) findViewById(R.id.my_button);
myButton.setOnClickListener((View v) -> {
	doAction(v);
});
\end{lstlisting}

\subsection*{How Does it Help Developers Write Better Code?}
Combining this Java feature with standard library code can provide developers a powerful syntax 
that is more concise and readable. For example you could have an interface such as this:
\begin{lstlisting}
interface Consumer {
	void accept(Object o);
}
\end{lstlisting}
further, for the {\ttfamily List} interface you had a method {\ttfamily void
forEach(Consumer c)}.  Following this, if you wanted to perform an action on
each element in a list you could use a lambda expression in the following way:
\begin{lstlisting}
// Print elements using lambda
listOfNumbers.forEach((Object o) -> System.out.println(o));

// Print elements using for loop
for (Object o : listOfNumbers) {
	System.out.println(o);
}
\end{lstlisting}
The first way of printing the listOfNumbers is a good example of passing a
behaviour as a parameter using
lambdas.\cite{website:why-we-need-lambda-expressions-in-java}\\

In general the use of lambdas as a sort hand or instantiating anonymous classes
that implement just one method reduces the amount of code developers have to
write, make it more concise and therefore increases the readablity of code.


\subsection*{Lambda Expression Implementation}
\subsubsection*{Symbol table}
For my symbol table, I have a Scope interface, a Symbol abstract class and an
abstract class ScopedSymbol that extends Scope. I have three classes that
extend ScopedSymbol. The first is JavaxSymbol that represents the global scope
of the javax file. There is no enclosing scope for the JavaxSymbol.
ClassSymbols are defined in JavaxSymbol's HashMap. ClassSymbols also have a
method for getting a list of methods that it contains. This is important for
figuring out what method lambda expressions implement.  MethodSymbols and
VariableSymbols are defined inside ClassSymbols.  MethodSymbols are also
ScopedSymbols. Therefore inside MethodSymbols VariableSymbols are defined.

A2Compiler runs the following visitors on the AST:
\begin{itemize}
\item CreateClassScopesVisitor
\item CheckClassExtendsAndImplementsVisitor
\item DefineMethodScopesVisitor
\item DefineVariableScopesVisitor
\item LambdaTypeResolverVisitor
\item CheckAssignmentTypesVisitor
\item DumbVisitor
\end{itemize}
CreateClassScopesVisitor initially defines the global scope, then goes on to
define all the primitive types and some standard library types. It then goes
through the AST visiting class definitions and defining them inside the global
scope.\\
CheckClassExtendsAndImplementsVisitor goes through the AST checking that the
classes defined in the javax file extend and implement classes that are in
scope.\\
DefineMethodScopesVisitor goes through the AST defining MethodSymbols in the
correct scope and checks that the method parameters and return type are in
scope.\\
DefineVariableScopesVisitor goes through and defines VariableSymbols the fields
and variables inside classes.\\
LambdaTypeResolverVisitor goes through the AST finding LambdaExpr nodes and
sets their data to be the class the lambda expression is implementing. This is
the most interesting visitor out of all the ones I implemented as it needs to
infer the type based on how the lambda is used. Therefore the lambda could be used as
a variable initialiser for a field or variable. It could also be used as an argument
to a method or it could be used as a return value. This visitor figures out the
type for all of those cases and adds that information to the LambdaExpr node
to be used later in the DumpVisitor.\\
CheckAssignmentTypesVisitor finds assignment expressions and checks that the
left sides type is compatible with the expression on the right side. This
visitor is restricted to primitive type literals due to time constraints.  So
this visitor will check to see if the left side is compatible with the
following literals IntegerLiteralExpr, DoubleLiteralExpr, BooleanLiteralExpr,
CharLiteralExpr and StringLiteralExpr.

\subsection*{Challenges I Faced}
The most challenging aspect of this assignment was to implement
LambdaTypeResolverVisitor, it required resolving the type of the lambda
expression in several use cases which were difficult to get right. It also
required me to make further modifications to the jj file after the initial
changes as I has missed some use cases of the lambda expression.


\bibliographystyle{IEEEtran}
\bibliography{references}
\end{document}
