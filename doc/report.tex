\documentclass[twocolumn]{report}
\usepackage{listings}
\usepackage{color}
\definecolor{javared}{rgb}{0.6,0,0} % for strings
\definecolor{javagreen}{rgb}{0.25,0.5,0.35} % comments
\definecolor{javapurple}{rgb}{0.5,0,0.35} % keywords
\definecolor{javadocblue}{rgb}{0.25,0.35,0.75} % javadoc
 
\lstset{language=Java,
basicstyle=\ttfamily\scriptsize,
keywordstyle=\color{javapurple}\bfseries,
stringstyle=\color{javared},
commentstyle=\color{javagreen},
morecomment=[s][\color{javadocblue}]{/**}{*/},
tabsize=4,
showspaces=false,
showstringspaces=false}

\begin{document}
\title{Java Lambda Expression Source to Source Compiler}
\author{Kurt McAlpine}
\date{24 April 2015}
\maketitle

\chapter{Introduction}
An example use of my implementation of lambda expressions in java is to
instantiate an anonymous class with exactly one method to be overriden. Using
lambda expressions it makes the code much cleaner and more concise.

For example this code:
\begin{lstlisting}
interface Addition {
  int sum(int a, int b);
}

public static void main(String[] args) {
  Addition addition = new Addition() {

    @Override
    public int sum(int a, int b) {
      return a + b;
    }
  };
  System.out.println(addition.sum(2, 5));
}
\end{lstlisting}
Would be written more consicely, like this:
\begin{lstlisting}
interface Addition {
  int sum(int a, int b);
}

public static void main(String[] args) {
  Addition addition = (int a, int b) -> {
    return a + b;
  };
  System.out.println(addition.sum(2, 5));
}
\end{lstlisting}
More concise:
\begin{lstlisting}
  Addition addition = (int a, int b) -> return a + b;
  System.out.println(addition.sum(2, 5));
\end{lstlisting}




\end{document}
