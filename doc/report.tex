\documentclass[twocolumn,notitlepage]{report}
\usepackage{hyperref}
\usepackage{listings}
\usepackage{color}
\usepackage{xcolor}
\definecolor{javared}{rgb}{0.6,0,0} % for strings
\definecolor{javagreen}{rgb}{0.25,0.5,0.35} % comments
\definecolor{javapurple}{rgb}{0.5,0,0.35} % keywords
\definecolor{javadocblue}{rgb}{0.25,0.35,0.75} % javadoc

% Change bibliography style
\usepackage{etoolbox}
\patchcmd{\thebibliography}{\chapter*}{\section*}{}{}
\renewcommand\bibname{References}

% No indentation
\setlength\parindent{0pt}
 
\lstset{language=Java,
breaklines=true,
basicstyle=\ttfamily\scriptsize,
keywordstyle=\color{javapurple}\bfseries,
stringstyle=\color{javared},
commentstyle=\color{javagreen},
morecomment=[s][\color{javadocblue}]{/**}{*/},
tabsize=4,
showspaces=false,
frame=lrtb,
showstringspaces=false}

\begin{document}
\title{Java Lambda Expression Source to Source Compiler}
\author{Kurt McAlpine}
\date{24 April 2015}
\maketitle

\section*{Introduction}
For this assignment I have decided to implement a feature similar to Java 8
lambda expressions. These expressions are used in place of anonymous class
instantiations.

An example use of my implementation of lambda expressions in java is to
instantiate an anonymous classes with exactly one method to be overridden. Using
lambda expressions it makes the code much cleaner and more
concise.

The following is an example use of my implementation of lambda expressions.

In the Java standard library there is this interface:
\begin{lstlisting}
interface Runnable {
  void run();
}
\end{lstlisting}

Instantiating an anonymous instance of Runnable:
\begin{lstlisting}
Runnable r = new Runnable() {
	@Override
	public void run() {
		System.out.println("run!");
	}
};
r.run();
\end{lstlisting}
Now using lambda expression:
\begin{lstlisting}
Runnable r = () -> { System.out.println("run!"); };    	
r.run();
\end{lstlisting}

The syntax of a lambda expression starts with a list of parameters in
parenthesis, for example: {\ttfamily(int a, int b, String s)} , followed by a
new keyword ``{\ttfamily->}'' and then a series of statements surrounded by
braces. In the about example only one statement is used but there may be
multiple statements inside the braces.

\section*{Why java lambda expressions are important}
Currently (in Java 1.7 and below) programmers need to write a large amount of
boiler plate code to instantiate an anonymous class. It takes around 3 lines to
create an anonymous instance of Runnable that overrides the run() method with a
method body of just one line. There is clearly a need for improvement.

\section*{How lambda expressions address the problematic issues that programmers face}
An example of a piece of java code that requires a lot of boilder plate code is
in Android developement, when you are attaching a click listener to a view such
as a button. So you may have some code that looks like this:
\begin{lstlisting}
Button myButton = (Button) findViewById(R.id.my_button);
myButton.setOnClickListener(
new View.OnClickListener() {
	@Override
	public void onClick(View v) {
		doAction(v);
	}
});
\end{lstlisting}
But using lambda expressions we can make this code less verbose and therefore
more readable and concise. Here is the above code rewritten to use lambda
expressions.
\newpage
\begin{lstlisting}
Button myButton = (Button) findViewById(R.id.my_button);
myButton.setOnClickListener((View v) -> {
	doAction(v);
});
\end{lstlisting}

\subsection*{How does it help developers write better code?}
Combining this Java feature with standard library code can provide developers a powerful syntax 
that is more concise and readable. For example you could have an interface such as this:
\begin{lstlisting}
interface Consumer {
	void accept(Object o);
}
\end{lstlisting}
And for the {\ttfamily List} interface you had a method {\ttfamily void
forEach(Consumer c)}.  And if you wanted to perform an action on each element
in a list you could use a lambda expression in the following way:
\begin{lstlisting}
// Print elements using lambda
listOfNumbers.forEach((Object o) -> System.out.println(o));

// Print elements using for loop
for (Object o : listOfNumbers) {
	System.out.println(o);
}
\end{lstlisting}
The first way of printing the listOfNumbers is a good example of passing a
behaviour as a paramter using lambdas.
\cite{website:why-we-need-lambda-expressions-in-java}


\subsection*{Lambda expression implementation}
\subsubsection*{Symbol table}
\subsubsection*{Which nodes this affects}
\subsection*{Challenges I faced}


\bibliographystyle{IEEEtran}
\bibliography{references}
\end{document}
